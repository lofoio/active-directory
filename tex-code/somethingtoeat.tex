\documentclass[unixfonts]{ltxdoc}
\usepackage[fntef,UTF8]{ctexcap}
\usepackage{texnames}
\usepackage[no-sscript]{xltxtra}
\usepackage{listings} % 程序源码格式输出
\usepackage{color} % 颜色相关
\usepackage[vmargin={0.5cm,0.5cm}]{geometry}
 \topmargin 0.5 true cm
 \oddsidemargin 1 true cm
 \evensidemargin 1 true cm
 \textheight 21 true cm
 \textwidth 14 true cm
%设定英文字体
\setmainfont{Nimbus Roman No9 L} %改成其他字体如 Times New Roman
%%  {\Huge \textbf{小学数学FAQ}}
\title{肉夹馍}
%\author{\href{mailto:roover.wd@gmail.com}{Wang Dian}}
\usepackage[colorlinks,bookmarks=false,pdfstartview=FitH,pdftitle=thoughtsofmime,pdfauthor=leafduo]{hyperref}

% 源代码显示配置
\lstloadlanguages{TEX}
\lstset{basicstyle=\linespread{1.0}\zihao{-6},breaklines,numbers=none,extendedchars=false,framexleftmargin=10mm,frame=single,backgroundcolor=\color[RGB]{245,245,244},keywordstyle=\bfseries\color[RGB]{130,0,0},identifierstyle=\bfseries\color[RGB]{0,0,130},numberstyle=\color[RGB]{41,41,255},commentstyle=\it\color[RGB]{130,130,130},stringstyle=\rmfamily\slshape\color[RGB]{255,0,0},showstringspaces=false,tabsize=4}
\thispagestyle{empty}
\begin{document}
\begin{center}
  {\huge \textbf{肉夹馍}}
\end{center}
\section{泡肉}
把5-10斤肉放入水中除去腥水,冬天用40度的温水,夏天用凉水。隔20分钟换次水,共泡1小时。

最好是猪前后腿肉,肥七瘦三。
\section{调料}
调料包可用两次,冷却后再放入冰箱,放在盘子上就行,不要罩盖子。

小香:一把,八角(八个角才是真的):七、八个,花椒:半把,白胡椒:20-30粒,肉蔻:一个(拍开),草寇:五个,桂丁:六、七个,香沙:三、四个,沙仁:十来个,良姜:一个,百里香:一手心/一撮,香叶:六、七片,白蔻:一手心,三奈:二片,甘草:一手心,生姜:五六片(小手指长),香果:半个/四分之一。

以下为中药,最好少放一点儿:


丁香:二个,桂皮:一拇指关节长,荜拨:一个,草果:一个(拍开),白芷:一片。


调料洗两三遍,如果是清香味则成功,如果是中药味则失败。
\section{熬肉}
把泡好的肉放入热锅中烫去毛,切成拇指粗细的块状,加入20斤水,20勺盐(二两左右),加入两小块冰糖(大拇指甲大小),一大勺糖色(一、二两),锅内水变红黑色。大火熬,烧开后加一两白酒去肉腥味。然后继续大火熬30分钟,中火熬30分钟,小火熬30分钟。然后提出调料包,若肉少可以早点儿提出调料包。每天熬开一次,可以长时间保存。消耗掉肉后可加入新肉,逐渐减少盐和糖色,每次减二勺左右,白酒和冰糖不变。

\section{白吉馍}
一斤面(高筋面),三克酵母,五克泡大粉,半斤四十度水,面要和硬点儿。和面时揉滑了就行了,然后用60度温水醒面,记住盖上保鲜膜。电饼档180度-200度。面饼要蓬松。擀面时要快,向后拉擀面杖要用力,一个饼擀三次。
\end{document}